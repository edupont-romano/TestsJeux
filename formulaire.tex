\documentclass[11pt,a4paper]{letter}
\usepackage{hyperref}
\usepackage{wasysym}
\usepackage[dvipsnames]{xcolor}
\definecolor{vert}{RGB}{0,155,0}
\definecolor{orange}{RGB}{255,100,0}
\usepackage[margin=0.2in]{geometry}
\pagestyle{empty}

\newcommand\checkbox[1]{%
    \Square\ #1\quad%
}
\newcommand{\note}[1]{
#1\\
\textcolor{red}{\checkbox{0}} \textcolor{red}{\checkbox{1}} \textcolor{red}{\checkbox{2}}  \textcolor{orange}	{\checkbox{3}}  \textcolor{orange}{\checkbox{4}}  \textcolor{orange}{\checkbox{5}} 	\checkbox{6} \checkbox{7} \textcolor{vert}{\checkbox{8}} \textcolor{vert}{\checkbox{9}} 	\textcolor{vert}{\checkbox{10}}\\
}

\newcommand{\points}{…………………………………………………………………………\\}
\newcommand{\pointsmultilines}{\\\points\points\points\points\points\points}

\begin{document}

Pour remplir ce questionnaire, merci de mettre une note de 0 à 10 à chaque question. 0 signifie \textit{Pas Du tout}, 10 signifie \textit{Complètement}.

Titre du jeu:\points
Version :\points
Email : \points
Nombre de joueurs : \points

\note{J'ai trouvé le titre pertinent:}
\note{J'ai aimé l'univers:}
\note{Le jeu est fluide:}
\note{J'ai eu plaisir à jouer:}
\note{J'ai envie d'y rejouer:}
\note{J'aurai envie de l'acheter:}
\note{J'ai envie de faire découvrir ce jeu autour de moi:}
\note{Je trouve la difficulté de ce jeu conforme à ce que j'attends:}
\note{Les règles sont lisibles:}
\note{Les règles sont claires:}
\note{Les tours ont une durée correcte:}
\note{Le temps entre les tours est correcte:}
\note{J'ai aimé le jeu:}
\note{Je trouve ce jeu original:}
\note{J'ai aimé l'interactivité entre les joueurs:}
J'ai aimé: \pointsmultilines
Je n'ai pas aimé:\pointsmultilines
Remarques:\pointsmultilines
\end{document}